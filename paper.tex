\documentclass{article}
%\documentclass[twocolumn]{article}%
\usepackage[utf8]{inputenc}%
\usepackage{standalone}%
\usepackage{amsmath}%
\usepackage{amsthm}%
\usepackage{amssymb}%
\usepackage{hyperref}%
\usepackage{xcolor}%
\usepackage{tabularx}%
\usepackage{listings}%
\usepackage{colortbl}
\usepackage{enumitem}

\usepackage{listings, xcolor}

\definecolor{verylightgray}{rgb}{.97,.97,.97}

\lstdefinelanguage{Solidity}{
	keywords=[1]{anonymous, assembly, assert, balance, break, call, callcode, case, catch, class, constant, continue, constructor, contract, debugger, default, delegatecall, delete, do, else, emit, event, experimental, export, external, false, finally, for, function, gas, if, implements, import, in, indexed, instanceof, interface, internal, is, length, library, log0, log1, log2, log3, log4, memory, modifier, new, payable, pragma, private, protected, public, pure, push, require, return, returns, revert, selfdestruct, send, solidity, storage, struct, suicide, super, switch, then, this, throw, transfer, true, try, typeof, using, value, view, while, with, addmod, ecrecover, keccak256, mulmod, ripemd160, sha256, sha3}, % generic keywords including crypto operations
	keywordstyle=[1]\color{blue}\bfseries,
	keywords=[2]{address, bool, byte, bytes, bytes1, bytes2, bytes3, bytes4, bytes5, bytes6, bytes7, bytes8, bytes9, bytes10, bytes11, bytes12, bytes13, bytes14, bytes15, bytes16, bytes17, bytes18, bytes19, bytes20, bytes21, bytes22, bytes23, bytes24, bytes25, bytes26, bytes27, bytes28, bytes29, bytes30, bytes31, bytes32, enum, int, int8, int16, int24, int32, int40, int48, int56, int64, int72, int80, int88, int96, int104, int112, int120, int128, int136, int144, int152, int160, int168, int176, int184, int192, int200, int208, int216, int224, int232, int240, int248, int256, mapping, string, uint, uint8, uint16, uint24, uint32, uint40, uint48, uint56, uint64, uint72, uint80, uint88, uint96, uint104, uint112, uint120, uint128, uint136, uint144, uint152, uint160, uint168, uint176, uint184, uint192, uint200, uint208, uint216, uint224, uint232, uint240, uint248, uint256, var, void, ether, finney, szabo, wei, days, hours, minutes, seconds, weeks, years},	% types; money and time units
	keywordstyle=[2]\color{teal}\bfseries,
	keywords=[3]{block, blockhash, coinbase, difficulty, gaslimit, number, timestamp, msg, data, gas, sender, sig, value, now, tx, gasprice, origin},	% environment variables
	keywordstyle=[3]\color{violet}\bfseries,
	identifierstyle=\color{black},
	sensitive=false,
	comment=[l]{//},
	morecomment=[s]{/*}{*/},
	commentstyle=\color{gray}\ttfamily,
	stringstyle=\color{red}\ttfamily,
	morestring=[b]',
	morestring=[b]"
}

\lstset{
	language=Solidity,
	backgroundcolor=\color{verylightgray},
	extendedchars=true,
	basicstyle=\footnotesize\ttfamily,
	showstringspaces=false,
	showspaces=false,
	numbers=left,
	numberstyle=\footnotesize,
	numbersep=9pt,
	tabsize=2,
	breaklines=true,
	showtabs=false,
	captionpos=b
}

\lstdefinelanguage{circom}{
	keywords=[1]{signal, input, output, public, template, component, var, function, return, if, else, for, while,
		do, log, assert, include, pragma, circom}, % generic keywords including crypto operations
	keywordstyle=[1]\color{blue}\bfseries,
	keywords=[2]{},	% types; money and time units
	keywordstyle=[2]\color{teal}\bfseries,
	keywords=[3]{},	% environment variables
	keywordstyle=[3]\color{violet}\bfseries,
	identifierstyle=\color{black},
	sensitive=false,
	comment=[l]{//},
	morecomment=[s]{/*}{*/},
	commentstyle=\color{gray}\ttfamily,
	stringstyle=\color{red}\ttfamily,
	morestring=[b]',
	morestring=[b]"
}

\lstset{
	language=circom,
	backgroundcolor=\color{verylightblue},
	extendedchars=true,
	basicstyle=\footnotesize\ttfamily,
	showstringspaces=false,
	showspaces=false,
	numbers=left,
	numberstyle=\footnotesize,
	numbersep=9pt,
	tabsize=2,
	breaklines=true,
	showtabs=false,
	captionpos=b
}

\newcommand{\khk}{K_{P}}
\newcommand{\khkp}{$K^{-1}[\text{HK}]$}
\newcommand{\kid}{K_{ID}}
%\newcommand{\kid}{$K$}
\newcommand{\todo}[1]{\fbox{\parbox{\linewidth}{\textcolor{red}{#1}}}}
\newcommand{\iv}{$IV(\text{\identity})$}
\newcommand{\sigid}{$K^{-1}[CA](K[Identity])$}
\newtheorem{problem}{Problem}
\newenvironment{solution}
  {\emph{Solution:}}
  {\hfill $\square$}
\newcommand{\pbc}{\textit{Identity Blockchain}}

\title{Identity Blockchain - Proof of Identity}
\author{
Albert Vučinović, University North\\
Miroslav Jerković, University of Zagreb}
\begin{document}
\maketitle

\begin{abstract}
$\pbc$ uses state-certified electronic identities (eIDs) to create blockchain identities and a consensus protocol called Proof of Identity (PoI). PoI is "green" and enables many applications that are impossible to implement without verified identity on the blockchain, e.g., direct democracy and universal basic income, encrypted messaging, etc. $\pbc$ preserves identity anonymity by using zk-SNARKs and an anonymization protocol for identity onboarding.
\end{abstract}

\section{Introduction}
TODO

\section{Notation}
\vspace{10px}
 \arrayrulecolor{gray}
\begin{tabular}{p{0.16\linewidth}| p{0.81\linewidth}}
  $(K, K^{-1})$ & \ \ Pair of public key $K$ and the corresponding private key $K^{-1}$ \\[3px]
	$K(value)$ & \ \ String $value$ encrypted with a public key $K$ \\[3px]
	$K^{-1}(value)$ & \ \ String $value$ encrypted with a private key $K^{-1}$ \\[3px]
  $H$ & \ \ zk-SNARK-friendly hash function \\[5px]
 \hline \\
   $Nin$ & \ \ Unique state-issued personal \textit{National Identification Number}\\[3px]
   $CA$ & \ \ Certificate Authority \\[3px]
   $eID$ & \ \ Electronic identification document \\[3px]
	$\kid$ & \ \ CA-certified public key of eID\\[3px]
  $\khk$ & \ \ Person key, unique public key used to access \pbc{}

\end{tabular}

\vspace{15px}

Sometimes we write $K$ for $(K, K^{-1})$.

\newpage
\section{Registering $\kid$}
Register $\kid$ on \pbc{} by calling $$register\kid(H(Nin), \kid^{-1}("register\kid"), proofNinOwnsKid).$$
\begin{enumerate}[leftmargin=0cm]
\item[] \underline{Pseudo-solidity code}

\hfill\begin{minipage}{\dimexpr\textwidth-20px}
\begin{lstlisting}[language=Solidity]
  mapping(uint256 => uint256) public ninKidUsed;
  mapping(uint256 => bool) public kidInvalid;

  function registerKid(
    uint256 hNin, // hNin = H(Nin)
    uint256 vKid, // vKid = KidPrivate("registerKid")
    bytes proofNinOwnsKid
  ) public {
    uint256 previousKey = ninKidUsed[hNin];
    if (previousKey == vKid) return; // already registered
    bool isOwner = zksnarkverify(VK, [hNin, vKid], proofNinOwnsKid); // VK = verification key
    if (isOwner && !kidInvalid[vKid]) {
      if (previousKey != 0) kidInvalid[previousKey] = true;
      ninKidUsed[hNin] = vKid;
    }
  }
\end{lstlisting}
\xdef\tpd{\the\prevdepth}
\end{minipage}

\item[] \underline{Pseudo-circom code}

\hfill\begin{minipage}{\dimexpr\textwidth-20px}
\begin{lstlisting}[language=circom]
pragma circom 2.0.0;

template NinOwnsKid(){
	//TODO: What happens to uint256 when converted to signal

	//public:
	signal input hNin;
	signal input vKid;

	//private:
	
	//Take Kid
	//Take CA

	
	
}
component main { public [hNin,vKid]} = NinOwnsKid();
\end{lstlisting}
\xdef\tpd{\the\prevdepth}
\end{minipage}

\item[] \underline{Features}
    \begin{itemize}
      \item[] \textbf{Sybil resistance} \\ A Person is prevented from registering more than one $\khk$, e.g. by using different identification documents issued for the same $Nin$.
          \vspace{5px}
      \item[] \textbf{Fix for the loss of eID} \\ A Person who has lost a previously registered $\kid$, e.g. due to loss of eID, can overwrite it by registering a new $\kid$.
          \vspace{5px}
      \item[] \textbf{Identity theft damage control} \\ If a Person overwrites the previous $\kid$ with a new one, the identity Thief loses access to services that verify the entire identity chain.
    \end{itemize}

\item[] \underline{Remarks}
\begin{itemize}
\item[i)] An entity that knows $\kid$, e.g. the CA who certified the corresponding eID, can know that the Person has registered $\kid$ on \pbc{}.
\item[ii)] Reasoning for the choice of arguments: if $H(Nin, \kid^{-1}("register\kid"))$ was used as an argument, then TODO. On the other hand, the choice of arguments $Nin$ and $\kid^{-1}("register\kid")$ would TODO.
\end{itemize}
\end{enumerate}

\newpage
\section{Registering $\khk$}
Register $\khk$ on \pbc{} by calling $$register\khk(Nin\kid, Nin\kid\khk, proof),$$
  where:
    \begin{enumerate}
      \item[] $Nin\kid=H(Nin,\kid^{-1}("register\khk"))$
      \item[] $Nin\kid\khk=H(Nin, \kid^{-1}("register\khk"), \khk^{-1}("register\khk"))$.
    \end{enumerate}
\begin{enumerate}[leftmargin=0cm]
\item[] \underline{Pseudo-solidity code}

\hfill\begin{minipage}{\dimexpr\textwidth-20px}
\begin{lstlisting}[language=Solidity]
  mapping(uint256 => uint256) public ninKidKpUsed;
  mapping(uint256 => bool) public kpInvalid;
  mapping(uint256 => uint256) public ninKidKpLastConfirmed;

  function registerKp(
    uint256 ninKid,
    uint256 ninKidKp,
    bytes proof
  ) public {
    uint256 previousKey = ninKidKpUsed[ninKid];
    if (previousKey == ninKid) return; // already registered
    bool ok = zksnarkverify(VK, [ninKid, ninKidKp], proof); // VK = verification key
    if (ok && !kpInvalid[ninKidKp]) {
      if (previousKey != 0) kpInvalid[previousKey] = true;
      ninKidKpUsed[ninKid] = ninKidKp;
      ninKidKpLastConfirmed[ninKid] = block.number;
    }
  }
\end{lstlisting}
\xdef\tpd{\the\prevdepth}
\end{minipage}
\item[] \underline{Features}
\begin{itemize}
      \item[] \textbf{Sybil resistance} \\ $\khk$ is a unique public key bound to the Person on the \pbc{}. By design, the Person cannot have two valid $\khk$ keys at the same time.

      TODO: Explanation/proof.
      \vspace{5px}
    \item[] \textbf{Anonimity} \\ No one who has a database with all public keys $\kid$ or/and public keys $\khk$ can tell, from the function call, which $Nin$, $\kid$ or $\khk$ was used. This is true under the assumption that CA, which certified the eID, does not have access to $\kid^{-1}$, i.e. that $\kid$ was securely generated on the eID.

    TODO: Explanation/proof.
\end{itemize}

\item[] \underline{Remarks}
\begin{itemize}
\item[i)] Reasoning behind the choice of arguments: TODO
\end{itemize}


\end{enumerate}

\newpage
\section{zkp code}
\begin{lstlisting}
//pseudo zokrates zk-SNARK
import "hashes/sha256/512bitPacked" as sha256packed
import "ecc/babyjubjubParams.code" as context
import "ecc/proofOfOwnership.code" as proofOfOwnership

def proofPINOwnsK(
  field[2] PINHash,
  field[2] K_1Hash,
  field[2]

  private field[?] CACert){
}

def proof_of_being_a_person(
  //publically known arguments
  field[2] PINKeyUsedHash,
  field[2] PINKeyPersonKeyIssuedHash,
  field[2] PersonKeyInvalidHash,
  field[2] KeyInvalidHash,

  //BC state
  field BC_PINUsed,//BC State
  field BC_PINKeyUsed,//BC State
  field BC_PINKeyPersonKeyIssued,//BC State
  field BC_PersonKeyInvalid,//BC State
  field BC_KeyInvalid,//BC State
  ?field BC_private_key_challenge?

  //CA keys!
  field[?][?] CAKeys,//? BC State
  ...

  //private data
  //15360 bit RSA key is equivalent to 256-bit symmetric keys
  //2048 bit RSA key is equivalent to 112 bit symmetric keys
  //eID has 2048 bit RSA
  private field PIN,//max 254 bits, using only 128 = 16 bytes
  private field K_private_bc_new,//ECC private key
  private field[?] K_new,//public RSA key from eID
  private field[?] K_1_new,//private RSA key from eID, actually signed message
  //K_1_new("PeopleBC person proof")
  private field K_private_bc_old,//ECC private key, old
  private field[?] K_1_old)//K_1_old("PeopleBC person proof")?
  //?private field[2] K_bc_new,//ECC public key ?no need for key pair verification?
  //?private field[2] K_bc_old,//ECC public key ?no need for key pair verification
  private field[?] CAcert,
  private field CAindex
{
  //actual checking code here
  field[2] hash_PIN = sha256packed([0,0,0,PIN])
  assert(hash_PIN == PINHash);//proving that we know the real PIN, unimportant

  //proving ownership of newly registered key,
  //no real need to prove ownership of bc key
  //we actually need to prove ownereship of K_1_new,
  //and its connection to PIN via CA
  context=context()//babyjubjubParams context
  proofOfOwnership(K_bc_new, K_private_bc_new, context)==1

  //prove that you own the K_new
  //pseudo
  RSADecrypt(K_new, K_1_new) == "PeopleBC Person Proof";//?+BC_private_key_challenge;

  //proove that Key is connected to CA
  //pseudo
  checkCASignature(CAKey[CAindex], CACert, K_new) == 1;

  //proove that K_new is PIN certificate
  //pseudo
  extractPIN(CACert)==PIN;

  //end

  //end..
}
\end{lstlisting}
\newpage
\section{Problems}
\begin{enumerate}[label=\textbf{P\arabic*}]
\item The rate of addition of identities, the problems of frequent changes of Merkle Tree Roots, and how long it takes to generate zkp proof, and block generation time.
\item \label{bigsteal} The Thief stole $\kid^{-1}$ before we registered on \pbc{} and then registered $\kid$ and $\khk$.
\item The Thief stole $\kid^{-1}$ after we registered on \pbc.
\item Trying to use unregistered Key.
\item Trying to register the same $Nin$ with multiple Keys (e.g. two physical ids).
\item \label{chain} Using $\khk$ without checking the whole $CA\rightarrow \kid \rightarrow \khk$ chain.
  \begin{itemize}
    \item[i)] Using $\khk$ to register as a mining key.
    \item[ii)] Invalidating $\khk$.
    \item[iii)] Using new $\khk$ as a mining key.
  \end{itemize}
  \underline{Solution}

  Look at the solution to \ref{miningKey}.

  The Mining Registry can accept new $\khk$ after an expiration period (which can be e.g. one day).
\item \label{miningKey} A combination of \ref{bigsteal} and \ref{chain}. The Thief steals $\kid$, registers $\khk$ and registers $\khk$ for mining. The problem is bigger, because we can't invalidate $\khk$ if we don't know which $\khk$ it is.

  \underline{Solution}

  We make $\khk$ renewable. We write the last block number it was renewed on to the \pbc.
  Mining Registry can choose to accept $\khk$ that is not older than some time period (for the Registry a good period would seem to be one day).
  When paying the miners, the Registry also requires proof of $\khk$.
\item Only $\khk$ is compromised.
\end{enumerate}
\end{document}
